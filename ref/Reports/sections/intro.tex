\section{Introduction}
In 2020 the world was hit by an, in recent times, unprecedented pandemic. Given the novelty of both this outbreak and the methods we have for measuring both the number of cases, severity of cases and other variables at any given time, this has opened an opportunity for research into the spread of COVID-19 and related diseases. 

The present study investigates the impact environmental factors, specifically the amount of precipitation in a given period, have on facilitating or restraining the spread of COVID-19 in the Netherlands \textbf{based on how they affect the number of hospitalized additions per region. We will often be referring to our dependent variable as cases per region for simplicity.}

We use the weather variables by cleaning the data and merging it together with the official statistics provided by the state's government regarding covid-19 so that we can calculate coefficients describing the relationship between environmental factors and covid-19 outbreaks.