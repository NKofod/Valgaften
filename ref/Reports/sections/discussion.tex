\section{Discussion}

The results for the different weather factors were somewhat contradictory.

The results of the Pearson correlation showed that the temperature above ground is the most statistically significant according to the Pearson correlation.

Whereas the OLS regression models including the regression merged with stringency index data show that total precipitation has the strongest correlation with total precipitation from the weather variables. The Bonferroni and Holm-Bonferroni returned "True" for total precipitation showcasing that it is indeed statistically significant with respect to cases per region. 

According to the analysis that we carried out on the provided data sets, total precipitation also has a strong correlation with our dependent variable.

A research paper \citep{Menebo2020} that researched temperature and precipitation associated with Covid-19, based in Oslo, Norway. The paper aimed at analyzing the correlation between weather and corona data. The results of their research showed that temperature and precipitation are associated with daily corona cases. This reflects our hypothesis. Temperature was positively associated with corona whereas precipitation is negatively related.

A hypothesis for the correlation of total precipitation to corona is when precipitation is high, people tend to stay inside due to the rain, snow, or other environmental factors while still going to work, school, or holding social gatherings indoors. By doing so, they tend to be close together with friends and family where air circulation is undesirable, meaning the spread of corona could be intensified.

Another hypothesis is that low precipitation enables people to spend more time outside allowing people to break restrictions and be exposed to the virus.

Both of the above hypotheses do not actually prove direct causation between the temperature above ground, precipitation, or other environmental factors and the spread of the virus. It can be noted that the environmental factors changing human behavior to that of them to having more social interactions carry an increased risk of virus transmission, but it does not actually create a biologically beneficial environment for the spread of the virus.
