\section{Data}
The data in the present investigation are from the following sources: 
\begin{table}[h!]
    \begin{tabular}{|p{.3\textwidth}|p{.6\textwidth}|}
        \hline 
        COVID-19 Data & "Novel Coronavirus (COVID-19) Cases in The Netherlands", De Bruin et el. \citep{DeBruin2020} \\ \hline 
        Population data & Statline, table 70072NED \\ \hline 
        Weather data & Provided by the course (BSFIYEP1KU at the IT University of Copenhagen) \\ \hline 
    \end{tabular}
\end{table}

\subsection{Data Cleaning}
\subsubsection{Initial Cleaning}
The data cleaning process can be described iteratively, since for both datasets used in the project the same process was used. 

\begin{enumerate}
    \item Load the data into a dataframe 
    \item Sanity checks 
    \begin{itemize}
        \item Check the dimensions of the data set 
        \item Check for missing values 
        \begin{itemize}
            \item If present - Remove rows with missing values
        \end{itemize}
        \item Check if each variable has been loaded in as the correct datatype 
        \begin{itemize}
            \item If not - Correct this 
        \end{itemize}
    \end{itemize}
    \item Calculate basic values 
    \begin{itemize}
        \item Mean 
        \item Median 
        \item Quartiles 
    \end{itemize}

\end{enumerate}

\subsubsection{Merging}
In order to actually investigate any potential correlations between the two datasets, we had to identify variables upon which to perform an outer join. In this case both datasets had a 'date'-column which, after the data cleaning, were in the same format. But given that for each day both data sets reported both national and regional numbers for the Netherlands, we also had to use this for the join, in order to get an accurate representation of any potential correlations. 

% Notes:

% corona.csv - everything with regards to covid based on regions including region_code, deceased_addition, confirmed_addition, hospitalized_addition, deceased_cumulative, confirmed_cumulative, hospitalized_cumulative where we focussed on hospitalized_addition.
% Metadata - Regions with populations, and its region code
% weather.csv - date, iso3166-2, RelativeHumiditySurface, SolarRadiation, Surfacepressure, TemperatureAboveGround, Totalprecipitation, UVIndex, WindSpeed. We focussed on total precipitaion. 

% Merged data sets

% MISSING: How we obtained, cleaned(without referring to code), How we dealt with issues such as missing data. 
% -Lost data by merging, and during a log transformation
% We dropped rows to remove NaNs
